\documentclass[a4paper,12pt]{article}
\usepackage[utf8]{inputenc}
\usepackage[spanish]{babel}
\usepackage{geometry}
\usepackage{graphicx}
\usepackage{amsmath}
\usepackage{hyperref}
\usepackage{longtable}
\usepackage{booktabs}
\usepackage{fancyhdr}
\usepackage{amsmath}

\usepackage{titling}
\pretitle{\begin{center}\Huge\bfseries}
\posttitle{\end{center}\vskip 0.5em}
\preauthor{\begin{center}\large}
\postauthor{\end{center}}
\predate{\begin{center}\large}
\postdate{\end{center}}

\geometry{top=2.5cm, bottom=2.5cm, left=2.5cm, right=2.5cm}

\title{Sistema de Evaluación Anticorrupción}
\author{Ricardo Emmanuel Uriegas Ibarra \\ 
\texttt{2230122@upv.edu.mx}}
\date{Noviembre 2024}

\begin{document}
% \begin{titlepage}
%     \centering
%     \vspace*{5cm}
%     {\Huge\bfseries Sistema de Evaluación Anticorrupción \par}
%     \vspace{1.5cm}
%     {\Large Ricardo Emmanuel Uriegas Ibarra \par}
%     \vspace{0.5cm}
%     {\large \texttt{2230122@upv.edu.mx} \par}
%     \vfill
%     {\large Noviembre 2024 \par}
% \end{titlepage}

\section{Resumen Ejecutivo}
Transparencia Internacional y Transparencia Mexicana presentaron el IPC 2023, donde México obtuvo 31/100 puntos, ubicándose en la posición 126 de 180 países. México es el último en la OCDE y penúltimo en el G20. Desde 2020, no ha mejorado su puntaje ni posición, y grandes casos de corrupción siguen sin resolverse. Con las próximas campañas presidenciales, la corrupción sigue siendo un reto crucial.

Proponemos una plataforma web modular que integra herramientas de análisis, auditorías y reportes para combatir la corrupción. Este sistema busca aumentar la transparencia, reducir los costos económicos asociados a la corrupción y fortalecer las instituciones mediante prácticas de rendición de cuentas. Las características principales incluyen monitoreo y auditoría de procesos, detección y prevención de prácticas corruptas, y fortalecimiento de capacidades institucionales.

\section{Antecedentes}
La corrupción es un problema persistente que afecta a diversas instituciones y sectores en todo el mundo. Según Transparencia Internacional, la corrupción socava el desarrollo económico, debilita la democracia y erosiona la confianza pública en las instituciones. Diversos estudios han demostrado que la corrupción tiene un impacto negativo significativo en el crecimiento económico y la equidad social.

\subsection{Fuentes de Consulta}
Para el desarrollo de este proyecto, se han consultado las siguientes fuentes:

\begin{itemize}
    \item Transparencia Internacional. (2023). \textit{Índice de Percepción de la Corrupción}. Recuperado de \url{https://www.transparency.org/en/cpi/2023}
    \item Banco Mundial. (2022). \textit{Informe sobre el Desarrollo Mundial: Gobernanza y la Ley}. Recuperado de \url{https://www.worldbank.org/en/publication/wdr2022}
    \item Transparencia Mexicana. (2023). \textit{La corrupción seguirá siendo un reto para el próximo gobierno}. Recuperado de \url{https://www.tm.org.mx/corrupcion-seguira-siendo-reto-para-el-proximo-gobierno-transparencia-mexicana/}
\end{itemize}

\textbf{Palabras Clave: }
Corrupción, Transparencia, Gobernanza, Auditoría, Análisis de Datos.

\section{Descripción de la Propuesta}
El sistema es una plataforma modular que integra herramientas de análisis, auditorías y reportes para abordar la corrupción desde múltiples frentes. Sus características principales incluyen:

\subsection{Justificación del Proyecto}
La corrupción afecta negativamente el desarrollo económico, la equidad social y la confianza en las instituciones. A pesar de los esfuerzos, persiste debido a su complejidad.

El Sistema de Evaluación Anticorrupción se justifica por:

\begin{itemize}
    \item \textbf{Transparencia:} Aumenta la transparencia en la gestión de recursos y decisiones, reduciendo oportunidades de corrupción.
    \item \textbf{Impacto Económico:} Detecta y previene prácticas corruptas, generando ahorros y mejorando la competitividad.
    \item \textbf{Fortalecimiento Institucional:} Mejora la capacidad de las instituciones para detectar y sancionar corrupción.
    \item \textbf{Innovación Tecnológica:} Utiliza análisis de datos y machine learning para identificar patrones y anomalías.
    \item \textbf{Cumplimiento Normativo:} Ayuda a cumplir regulaciones de transparencia y anticorrupción.
\end{itemize}

\subsection{Novedad o Innovación del Proyecto}
El Sistema de Evaluación Anticorrupción presenta varias innovaciones que lo distinguen de otras soluciones existentes en el mercado:

\begin{itemize}
    \item \textbf{Integración de Tecnologías Avanzadas:} Utiliza análisis de datos, machine learning y auditorías automatizadas para detectar patrones y anomalías que podrían pasar desapercibidos en auditorías tradicionales.
    \item \textbf{Plataforma Modular:} La arquitectura modular permite la personalización y escalabilidad del sistema, adaptándose a las necesidades específicas de diferentes instituciones.
\end{itemize}

\subsection{Similitudes}
El Sistema de Evaluación Anticorrupción comparte algunas características con otras soluciones disponibles en el mercado:

\begin{itemize}
    \item \textbf{Auditorías y Monitoreo:} Al igual que otros sistemas, incluye herramientas para la auditoría y el monitoreo de procesos.
    \item \textbf{Cumplimiento Normativo:} Ayuda a las organizaciones a cumplir con las regulaciones y normativas en materia de transparencia y anticorrupción.
    \item \textbf{Análisis de Datos:} Utiliza técnicas de análisis de datos para identificar irregularidades y patrones sospechosos.
\end{itemize}

\subsection{Diferencias}
El Sistema de Evaluación Anticorrupción se diferencia de otras soluciones en varios aspectos clave:

\begin{itemize}
    \item \textbf{Enfoque Integral:} Combina múltiples herramientas y tecnologías en una plataforma unificada para abordar la corrupción desde diferentes ángulos.
    \item \textbf{Personalización y Escalabilidad:} La arquitectura modular permite personalizar y escalar el sistema según las necesidades específicas de cada institución.
    \item \textbf{Tecnologías Avanzadas:} Incorpora machine learning y auditorías automatizadas, lo que representa una innovación significativa en la detección de patrones y anomalías.
    \item \textbf{Accesibilidad y Seguridad:} Ofrece acceso remoto seguro a través de la nube y utiliza cifrado avanzado para proteger la información sensible.
    \item \textbf{Reportes Automatizados:} Genera reportes personalizados y automatizados que facilitan la toma de decisiones informadas y oportunas.
\end{itemize}

\subsection{Módulos del Sistema}
\begin{enumerate}
    \item \textbf{Auditorías Cruzadas:} Comparación entre registros de empresas y gobiernos para identificar inconsistencias.
    \item \textbf{Análisis Predictivo:} Algoritmos entrenados para detectar patrones irregulares.
    \item \textbf{Reportes Automatizados:} Generación de métricas personalizadas.
    \item \textbf{Accesibilidad en la Nube:} Infraestructura escalable que permite el acceso remoto.
    \item \textbf{Seguridad de Datos:} Implementación de cifrado avanzado para proteger información sensible.
\end{enumerate}

\subsection{Flujo Operativo}
El flujo de operación del sistema incluye los siguientes pasos:
\begin{enumerate}
    \item Recolección de datos de las instituciones participantes.
    \item Análisis preliminar para identificar posibles anomalías.
    \item Implementación de auditorías cruzadas y generación de alertas.
    \item Entrega de reportes a usuarios clave.
\end{enumerate}

\section{Definición del Negocio}

\begin{itemize}
    \item \textbf{Transparencia:} Monitorea y audita procesos, aumentando la rendición de cuentas.
    \item \textbf{Seguridad:} Implementa cifrado avanzado para proteger datos sensibles.
    \item \textbf{Accesibilidad:} Plataforma modular y en la nube, personalizable y escalable.
    \item \textbf{Reportes:} Genera reportes automatizados para decisiones informadas.
\end{itemize}

% El negocio se enfoca en proporcionar una solución tecnológica que no solo detecte y prevenga la corrupción, sino que también promueva una cultura de transparencia y responsabilidad en las instituciones. La combinación de tecnologías avanzadas y una plataforma modular permite ofrecer un producto adaptable y escalable, adecuado para diferentes tipos de organizaciones, desde pequeñas empresas hasta grandes entidades gubernamentales.

\section{Impacto Social (Empresa Socialmente Responsable ESR)}
A continuación, se detallan los principales aspectos del impacto social del proyecto:

\subsection{Promoción de la Transparencia}
El sistema fomenta la transparencia y rendición de cuentas en instituciones, reduciendo prácticas corruptas y aumentando la confianza pública.

\subsection{Mejora de la Gobernanza}
Fortalece la capacidad institucional para detectar y sancionar corrupción, mejorando la gobernanza y la calidad de los servicios.

\subsection{Impacto Económico Positivo}
Reduce la corrupción, mejorando la eficiencia y competitividad económica, generando ahorros y beneficios para la sociedad.

\subsection{Cumplimiento de Normativas}
Ayuda a cumplir regulaciones de transparencia y anticorrupción, evitando sanciones y mejorando la reputación institucional.

\subsection{Innovación y Desarrollo Tecnológico}
Utiliza análisis de datos y machine learning, mejorando auditorías y promoviendo la adopción de nuevas tecnologías.

\subsection{Responsabilidad Social Corporativa}
Promueve prácticas éticas y sostenibles, demostrando el compromiso institucional con la transparencia y la responsabilidad social.


\section{Análisis de Mercado}
\subsection{Mercado Meta}
\begin{itemize}
    \item \textbf{Empresas Privadas:} Buscan optimizar recursos y minimizar riesgos de corrupción.
    \item \textbf{Gobiernos:} Necesitan soluciones robustas para gestión y análisis de datos.
\end{itemize}

\subsection{Tamaño del Mercado Meta}
El mercado meta se estima en función de la cantidad de organizaciones y su capacidad de inversión. Se proyecta un crecimiento anual del \(10\%\) en la demanda de estas soluciones:

\begin{itemize}
    \item \textbf{Empresas Privadas:} 100,000 empresas con un presupuesto promedio de \$50,000 anuales.
    \item \textbf{Gobiernos:} 10,000 instituciones con un presupuesto promedio de \$100,000 anuales.
\end{itemize}

\subsection{Datos Demográficos}
Incluyen:

\begin{itemize}
    \item \textbf{Ubicación Geográfica:} América del Norte, Europa y Asia-Pacífico.
    \item \textbf{Tamaño de la Empresa:} Desde pequeñas empresas hasta grandes corporaciones.
    \item \textbf{Sector Industrial:} Finanzas, gobierno, salud, tecnología y manufactura.
\end{itemize}

\subsubsection{Análisis de Competencia}
\textbf{Competidor: ACL (Galvanize)}

\begin{itemize}
    \item \textbf{Fortalezas:}
    \begin{itemize}
        \item Experiencia en auditoría y gestión de riesgos.
        \item Plataforma robusta con análisis de datos y machine learning.
        \item Presencia internacional y base de clientes diversificada.
    \end{itemize}
    \item \textbf{Debilidades:}
    \begin{itemize}
        \item Costos elevados.
        \item Complejidad en implementación y uso.
    \end{itemize}
    \item \textbf{Oportunidades:}
    \begin{itemize}
        \item Expansión en mercados emergentes.
        \item Integración de nuevas tecnologías.
    \end{itemize}
    \item \textbf{Amenazas:}
    \begin{itemize}
        \item Competencia de nuevas empresas tecnológicas.
        \item Cambios en regulaciones y políticas.
    \end{itemize}
\end{itemize}

% Este análisis nos permite identificar áreas donde nuestro sistema puede diferenciarse y ofrecer ventajas competitivas, como precios más accesibles, facilidad de uso y soporte técnico personalizado.
\subsection{Análisis PEST (Político/Legales, Económico, Social/Cultural y Tecnológico)}
El análisis PEST evalúa los factores Políticos/Legales, Económicos, Sociales/Culturales y Tecnológicos que afectan el mercado:

\begin{itemize}
    \item \textbf{Político/Legales:}
    \begin{itemize}
        \item Regulaciones sobre protección de datos y privacidad.
        \item Políticas gubernamentales de adopción tecnológica.
        \item Normativas anticorrupción y de transparencia.
    \end{itemize}
    \item \textbf{Económicos:}
    \begin{itemize}
        \item Crecimiento económico y capacidad de inversión de las empresas.
        \item Fluctuaciones en los costos de infraestructura tecnológica.
        \item Disponibilidad de financiamiento para proyectos tecnológicos.
    \end{itemize}
    \item \textbf{Social/Culturales:}
    \begin{itemize}
        \item Aceptación y adopción de nuevas tecnologías por parte de las organizaciones.
        \item Necesidad creciente de transparencia y seguridad en la gestión de datos.
        \item Conciencia social sobre la importancia de combatir la corrupción.
    \end{itemize}
    \item \textbf{Tecnológicos:}
    \begin{itemize}
        \item Avances en tecnologías de análisis de datos y machine learning.
        \item Disponibilidad de infraestructura en la nube y soluciones escalables.
        \item Innovaciones en ciberseguridad y protección de datos.
    \end{itemize}
\end{itemize}

\section{Factibilidad}
\subsection{Factibilidad Técnica (Maquinaria, Equipos, Tecnología, Conocimientos)}
% El Sistema de Evaluación Anticorrupción requiere una infraestructura tecnológica robusta y conocimientos especializados para su implementación y operación. A continuación se detallan los aspectos técnicos necesarios:

\begin{itemize}
    \item \textbf{SEO:} Mejorar el posicionamiento web para atraer tráfico orgánico.
    \item \textbf{Publicidad en Línea:} Anuncios pagados en Google Ads y redes sociales.
    \item \textbf{Marketing de Contenidos:} Crear materiales que destaquen beneficios y casos de éxito.
\end{itemize}

\subsection{Factibilidad Operativa (Instalaciones, Proveedores de Insumos, Repuestos y Servicios)}
Para asegurar la operatividad del sistema, se deben considerar los siguientes aspectos:

\begin{itemize}
    \item \textbf{Instalaciones:} Espacios físicos adecuados para alojar los equipos y servidores necesarios. 
    \item \textbf{Proveedores de Insumos:} Identificación de proveedores confiables para el suministro de hardware, software y servicios en la nube.
    \item \textbf{Repuestos y Servicios:} Disponibilidad de repuestos y servicios de mantenimiento para asegurar la continuidad operativa del sistema.
\end{itemize}

% \subsubsection{Cronograma de Actividades (Diagrama de Gantt Mensual)}
\subsubsection{Cronograma de Actividades}
% El siguiente cronograma detalla las actividades planificadas para el desarrollo y lanzamiento del Sistema de Evaluación Anticorrupción:

\begin{itemize}
    \item \textbf{Análisis de Requerimientos:} 1 mes
    \item \textbf{Diseño del Sistema:} 2 meses
    \item \textbf{Desarrollo del Sistema:} 4 meses
    \item \textbf{Pruebas y Validación:} 1.5 meses
    \item \textbf{Implementación y Despliegue:} 1 mes
    \item \textbf{Capacitación de Usuarios:} 0.5 meses
    \item \textbf{Lanzamiento del Sistema:} 1 día
\end{itemize}

\subsection{Factibilidad Económica}
\subsubsection{Inversión Inicial}
\begin{itemize}
    \item \textbf{Hardware:} \$100,000 MXN
    \item \textbf{Software:} \$50,000 MXN
    \item \textbf{Infraestructura en la Nube:} \$30,000 MXN
    \item \textbf{Otros Recursos:} \$20,000 MXN
\end{itemize}

El costo total inicial (\(C_i\)) se calcula como:
\[
C_i = \text{Hardware} + \text{Software} + \text{Infraestructura en la Nube} + \text{Otros Recursos}
\]
\[
C_i = 100,000 + 50,000 + 30,000 + 20,000 = 200,000 \, \text{MXN}
\]

\subsubsection{Fuentes de Financiamiento}
Las fuentes de financiamiento para el proyecto pueden incluir:

\begin{itemize}
    \item \textbf{Inversores Privados:} Capital de riesgo y fondos de inversión interesados en proyectos tecnológicos innovadores.
    \item \textbf{Subvenciones Gubernamentales:} Programas de apoyo a la innovación y la transparencia.
    \item \textbf{Préstamos Bancarios:} Créditos específicos para el desarrollo de proyectos tecnológicos.
\end{itemize}

\subsubsection{Rentabilidad}
\textbf{7.3.3.1 La TIR (Tasa Interna de Retorno)}
La inversión inicial es de \$21,000 MXN. La TIR es la tasa \( r \) que satisface la siguiente ecuación:

\[
- \$21,000 + \sum_{t=1}^{5} \frac{\text{Flujo de Caja}_t}{(1 + r)^t} = 0
\]

Calculando numéricamente, obtenemos una TIR aproximada de \textbf{15.24\%}.

\textbf{7.3.3.2 El VPN (Valor Presente Neto)}
Utilizando una tasa de descuento del \textbf{12\%}, el VPN se calcula de la siguiente manera:

\begin{align*}
    \text{VPN} \quad &= -\$21,000 + \sum_{t=1}^{5} \frac{\text{Flujo de Caja}_t}{(1 + 0.12)^t} \\
    &= -\$21,000 + \left( \frac{4,500}{1.12^1} + \frac{5,000}{1.12^2} + \frac{5,500}{1.12^3} + \frac{6,000}{1.12^4} + \frac{6,500}{1.12^5} \right) \\
    &= -\$21,000 + (\$4,017.86 + \$3,985.00 + \$3,909.28 + \$3,794.15 + \$3,643.54) \\
    &= -\$21,000 + \$19,349.83 \\
    &= \textbf{\$-1,650.17}
\end{align*}

El \textbf{VPN es de -\$1,650.17 MXN}, lo que indica que, bajo estas condiciones, el proyecto no recupera completamente la inversión inicial cuando se descuenta al 12\%.

\textbf{7.3.3.3 El ROI (Retorno Sobre la Inversión)}
La ganancia neta acumulada es:

\[
\text{Ganancia Neta} = (\$4,500 + \$5,000 + \$5,500 + \$6,000 + \$6,500) - \$21,000 = \$6,500
\]

El ROI es:

\[
\text{ROI} = \left( \frac{\$6,500}{\$21,000} \right) \times 100\% = \textbf{30.95\%}
\]

\textbf{7.3.3.4 Análisis de Escenarios}

\textbf{Escenario Optimista:} Incremento del 5\% en los flujos de caja. Calculando el VPN optimista:

\[
\text{VPN}_{\text{optimista}} = -\$21,000 + \sum_{t=1}^{5} \frac{\text{Flujo Optimista}_t}{(1 + 0.12)^t} \approx \$97.17
\]

El VPN optimista es \textbf{\$97.17 MXN}, lo que indica que el proyecto es rentable bajo este escenario.

\textbf{Escenario Pesimista:} Disminución del 5\% en los flujos de caja. Calculando el VPN pesimista:

\[
\text{VPN}_{\text{pesimista}} = -\$21,000 + \sum_{t=1}^{5} \frac{\text{Flujo Pesimista}_t}{(1 + 0.12)^t} \approx \$-3,397.51
\]

El VPN pesimista es \textbf{\$-3,397.51 MXN}, indicando que el proyecto no es rentable bajo este escenario.

\textbf{Conclusión:} Con una inversión inicial de \$21,000 MXN, el proyecto es marginalmente rentable.


\subsection{Resultados Esperados a Mediano y Largo Plazo}
A mediano y largo plazo, se esperan los siguientes resultados:

\begin{itemize}
    \item \textbf{Expansión del Mercado:} Implementación en más instituciones, tanto públicas como privadas, a nivel nacional e internacional.
    \item \textbf{Innovación Continua:} Desarrollo de nuevas funcionalidades aprovechando avances tecnológicos.
    \item \textbf{Sostenibilidad Financiera:} Generación de ingresos suficientes para cubrir costos operativos y permitir reinversión.
\end{itemize}

\section{Estrategia de Comercialización}
\subsection{Canales de Distribución}
Para asegurar una distribución efectiva del Sistema de Evaluación Anticorrupción, se utilizarán los siguientes canales:

\begin{itemize}
    \item \textbf{Distribución Directa} 
    \item \textbf{Distribución en Línea}
    \item \textbf{Alianzas Estratégicas}
\end{itemize}

\subsection{Plan de Marketing}
El plan de marketing se centrará en aumentar la visibilidad del sistema y atraer a clientes potenciales a través de diversas estrategias:

\subsubsection{Productos o Servicios}
El Sistema de Evaluación Anticorrupción se ofrecerá en diferentes paquetes adaptados a las necesidades de cada tipo de cliente:

\begin{itemize}
    \item \textbf{Paquete Básico:} \$50,000 MXN mensuales
    \item \textbf{Paquete Avanzado:} \$150,000 MXN mensuales
    \item \textbf{Paquete Premium:} \$300,000 MXN mensuales
\end{itemize}

\subsubsection{Objetivos}
Los objetivos del plan de marketing incluyen:

\begin{itemize}
    \item \textbf{Aumentar la Visibilidad:} Incrementar el conocimiento del sistema en el mercado objetivo.
    \item \textbf{Generar Clientes Potenciales:} Atraer y convertir clientes potenciales a través de campañas de marketing digital.
    \item \textbf{Fidelización de Clientes:} Mantener y fortalecer las relaciones con los clientes actuales mediante soporte continuo y actualizaciones del sistema.
\end{itemize}

\subsubsection{Medios de Comunicación}
Se utilizarán los siguientes medios de comunicación:

\begin{itemize}
    \item \textbf{Sitio Web:} Información detallada, casos de éxito y testimonios.
    \item \textbf{Redes Sociales:} Campañas en LinkedIn, Twitter y Facebook.
    \item \textbf{Email Marketing:} Boletines y promociones a clientes potenciales.
    \item \textbf{Webinars y Conferencias:} Presentaciones en línea y eventos en vivo.
\end{itemize}

\subsubsection{Marketing}
Las estrategias de marketing incluirán:

\begin{itemize}
    \item \textbf{SEO (Optimización para Motores de Búsqueda):} Mejorar el posicionamiento del sitio web en los resultados de búsqueda para atraer tráfico orgánico.
    \item \textbf{Publicidad en Línea:} Anuncios pagados en Google Ads y redes sociales para aumentar la visibilidad y atraer leads.
    \item \textbf{Marketing de Contenidos:} Creación de materiales que destaquen los beneficios y casos de éxito del sistema.
\end{itemize}

\section{Conclusión}
El Sistema de Evaluación Anticorrupción ofrece una solución innovadora para combatir la corrupción en instituciones. Integrando análisis de datos, machine learning y auditorías automatizadas, proporciona una plataforma adaptable a diversas organizaciones.

La demanda de soluciones que mejoren la transparencia y eficiencia es creciente. Nuestro sistema se posiciona como líder en la lucha contra la corrupción.

La evaluación técnica y económica indica una inversión recuperable a corto plazo y alto potencial de crecimiento. La estrategia de comercialización y marketing digital asegurará una amplia adopción en el mercado.

\end{document}