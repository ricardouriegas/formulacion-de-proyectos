\documentclass[a4paper,12pt]{article}
\usepackage[utf8]{inputenc}
\usepackage{graphicx}
\usepackage{geometry}
\usepackage{amsmath}
\usepackage{hyperref}
\usepackage{lipsum} % Solo para llenar con texto de ejemplo

\geometry{top=2.5cm, bottom=2.5cm, left=2.5cm, right=2.5cm}

\title{Proyecto de Evaluación Anticorrupción}
\author{Ricardo Emmanuel Uriegas Ibarra}
\date{Noviembre 2024}

\begin{document}

\maketitle

% \tableofcontents
% \newpage
\section{Resumen ejecutivo}
El presente documento propone un sistema tecnológico centralizado para la mitigación de la corrupción, enfocado en la transparencia y la colaboración entre empresas y gobiernos. Este sistema integra tecnologías modernas como aprendizaje automático y ciencia de datos para analizar patrones de corrupción y fomentar una gestión ética y eficiente.

\section{Antecedentes}
La corrupción representa un desafío crítico para los gobiernos y las instituciones públicas y privadas. Diversos informes de transparencia han señalado la necesidad de sistemas tecnológicos avanzados para prevenir, detectar y combatir prácticas corruptas (\cite{transparency2020}; \cite{ocde2021}).

En los últimos años, la adopción de tecnologías emergentes ha transformado sectores clave. La aplicación de \textbf{Machine Learning} y \textbf{ciencia de datos} ha permitido a las organizaciones identificar irregularidades financieras y operativas de manera proactiva. Sin embargo, la integración de estas tecnologías en plataformas anticorrupción aún es limitada, lo que deja espacio para innovaciones centradas en la colaboración interinstitucional (\cite{morrison2022}; \cite{smith2023}).

Un sistema centralizado para la prevención de la corrupción permitiría una mayor visibilidad de las operaciones, promoviendo la ética y la transparencia en todas las transacciones entre empresas y gobiernos.

\textbf{Palabras clave}: anticorrupción, transparencia, tecnología, colaboración.

\section{Descripción de la propuesta}
La propuesta consiste en desarrollar un sistema tecnológico que permita identificar, monitorear y reportar actividades potencialmente corruptas mediante análisis de datos y algoritmos de inteligencia artificial. Este sistema facilitará la colaboración entre instituciones públicas y privadas, permitiendo un enfoque preventivo y proactivo en la gestión de recursos.

\subsection{Justificación del proyecto}
La corrupción reduce la confianza en las instituciones y genera pérdidas significativas para la economía. Este proyecto responde a la necesidad de implementar una solución tecnológica que detecte prácticas corruptas y promueva la transparencia. Al centralizar la información y automatizar su análisis, se busca fortalecer la confianza ciudadana y reducir los riesgos asociados.

\subsection{Novedad o innovación del proyecto}
El sistema es innovador porque combina:
\begin{itemize}
    \item \textbf{Centralización}: Todas las operaciones relacionadas se gestionan en un solo sistema, facilitando la supervisión y el análisis.
    \item \textbf{Inteligencia artificial}: Detecta patrones anómalos en tiempo real.
    \item \textbf{Colaboración interinstitucional}: Promueve la transparencia y la rendición de cuentas entre empresas y gobiernos.
\end{itemize}

\subsection{Similitudes}
Comparte similitudes con plataformas de gestión de riesgos y transparencia, donde se promueve la integridad y la rendición de cuentas.

\subsection{Diferencias}
A diferencia de otras plataformas, este sistema se enfoca en la colaboración directa y el análisis predictivo mediante inteligencia artificial, integrando información de múltiples fuentes.

\section{Definición del negocio}
El sistema es una plataforma tecnológica que permite identificar y mitigar riesgos de corrupción mediante el análisis de datos y la colaboración interinstitucional.

\subsection{Propuesta de valor del negocio}
La propuesta de valor radica en un enfoque preventivo que utiliza tecnología avanzada para garantizar la transparencia y promover la confianza en las operaciones entre empresas y gobiernos.

\section{Impacto social}
Este sistema mejora la transparencia, fomenta la ética en las operaciones institucionales y fortalece la confianza pública en las instituciones.

\section{Análisis de Mercado}
\subsection{Mercado meta}
El mercado incluye:
\begin{itemize}
    \item Instituciones gubernamentales.
    \item Empresas privadas interesadas en garantizar prácticas éticas.
    \item Organizaciones no gubernamentales enfocadas en transparencia.
\end{itemize}

\subsection{Tamaño del mercado meta}
El tamaño del mercado meta incluye gobiernos y empresas a nivel nacional e internacional, especialmente en países con altos índices de corrupción.

\subsection{Datos demográficos}
El público objetivo son instituciones públicas y privadas con interés en mejorar la transparencia y la gestión de recursos.

\subsubsection{Análisis de la competencia}
Existen sistemas de auditoría y gestión de riesgos, pero pocos integran inteligencia artificial y colaboración interinstitucional como eje principal.

\subsection{Análisis PEST}
\begin{itemize}
    \item \textbf{Político/Legal:} Cumplimiento de leyes anticorrupción nacionales e internacionales.
    \item \textbf{Económico:} Reducción de costos asociados a pérdidas por corrupción.
    \item \textbf{Social/Cultural:} Aumento de la presión social por mayor transparencia.
    \item \textbf{Tecnológico:} Avances en inteligencia artificial y ciencia de datos.
\end{itemize}

\section{Factibilidad}

\subsection{Factibilidad técnica}
El sistema utiliza tecnologías como \textbf{Python} y \textbf{TensorFlow}, bases de datos como \textbf{PostgreSQL}, y servidores en la nube para escalabilidad.

\subsection{Factibilidad operativa}
El sistema operará en la nube, garantizando disponibilidad, seguridad y escalabilidad.

\subsubsection{Cronograma de actividades}
Las actividades incluyen el diseño, desarrollo, pruebas y despliegue del sistema.

\subsection{Factibilidad económica}
Los ingresos provendrán de:
\begin{itemize}
    \item Suscripciones al sistema.
    \item Servicios adicionales como auditorías personalizadas.
\end{itemize}

\subsubsection{Inversión inicial}
La inversión inicial incluye desarrollo, infraestructura y campañas de lanzamiento.

\subsubsection{Fuentes de financiamiento}
Se financiará mediante inversión privada y apoyo gubernamental.

\subsubsection{Rentabilidad}
El modelo de rentabilidad del proyecto se basa en su capacidad para generar ingresos sostenibles a través de sus principales fuentes: publicidad digital, comisiones por donaciones y tarifas por campañas destacadas. La publicidad, como principal motor de ingresos, permitirá cubrir los costos operativos a medida que la plataforma gane tráfico. Las comisiones del \textbf{5\%} por donaciones procesadas y las tarifas por campañas destacadas complementarán los ingresos, garantizando un flujo constante.

La plataforma está diseñada para escalar, incrementando sus ingresos proporcionalmente al número de usuarios y fundaciones participantes. Este modelo asegura que, tras el periodo de incubación, los ingresos superen los costos operativos, logrando la autosuficiencia financiera en el mediano plazo.

\textbf{La TIR}
La Tasa Interna de Retorno (TIR) se calcula considerando una inversión inicial de \textbf{10,000 pesos mexicanos} y los ingresos proyectados durante el periodo de incubación (10 meses) y los dos años posteriores.

Proyección de ingresos:
\begin{itemize}\setlength{\itemsep}{0pt}\setlength{\parskip}{1pt}
    \item \textbf{Periodo de incubación (10 meses):} 18,000 pesos, distribuidos en:
    \begin{itemize}\setlength{\itemsep}{0pt}\setlength{\parskip}{0pt}
        \item \textbf{Publicidad digital:} 12,000 pesos.
        \item \textbf{Comisiones por donaciones:} 6,000 pesos (5\% de comisión promedio).
    \end{itemize}
    \item \textbf{Año 2:} 30,000 pesos.
    \item \textbf{Año 3:} 40,000 pesos.
\end{itemize}

La TIR se define como la tasa de descuento que hace que el Valor Presente Neto (VPN) sea igual a cero. La ecuación a resolver es:

\[
\text{VPN} = -10,000 + \frac{18,000}{(1 + \text{TIR})^1} + \frac{30,000}{(1 + \text{TIR})^2} + \frac{40,000}{(1 + \text{TIR})^3} = 0
\]

Resolviendo esta ecuación, se obtiene un valor de TIR aproximado de:

\[
\textbf{36.5\%}.
\]

Este resultado muestra una alta rentabilidad del proyecto, demostrando que es financieramente viable y tiene un potencial significativo de crecimiento en el corto y mediano plazo.

\textbf{El VPN}
El Valor Presente Neto (VPN) se calcula considerando una inversión inicial de \textbf{10,000 pesos mexicanos} y los ingresos proyectados durante tres años:
\begin{itemize}\setlength{\itemsep}{0pt}\setlength{\parskip}{1pt}
    \item \textbf{Año 1:} 18,000 pesos.
    \item \textbf{Año 2:} 30,000 pesos.
    \item \textbf{Año 3:} 40,000 pesos.
\end{itemize}

Usando una tasa de descuento del \textbf{8\%}, la fórmula es:
\[
VPN = \frac{18,000}{1.08} + \frac{30,000}{1.08^2} + \frac{40,000}{1.08^3} - 10,000
\]

Cálculo:
\[
VPN = 16,667 + 25,718 + 31,749 - 10,000 = \textbf{64,134 pesos}.
\]

El resultado positivo muestra la viabilidad económica del proyecto y su capacidad de generar retornos por encima de la inversión inicial.

\textbf{El ROI}
El Retorno sobre la Inversión (ROI) se calcula considerando una inversión inicial de \textbf{10,000 pesos mexicanos} y un total de ingresos proyectados de \textbf{88,000 pesos} en los primeros tres años. 

Fórmula:
\[
ROI = \frac{\text{Ingresos totales} - \text{Inversión inicial}}{\text{Inversión inicial}} \times 100
\]

Cálculo:
\[
ROI = \frac{88,000 - 10,000}{10,000} \times 100 = \textbf{780\%}.
\]

Este ROI demuestra una alta rentabilidad, asegurando un retorno significativo sobre la inversión inicial.

\textbf{Análisis de sensibilidad}
El análisis de escenarios evalúa la viabilidad del proyecto considerando diferentes condiciones de mercado:

\begin{itemize}\setlength{\itemsep}{0pt}\setlength{\parskip}{1pt}
    \item \textbf{Escenario optimista:} Un rápido crecimiento de usuarios y campañas genera ingresos anuales de \textbf{40,000 pesos}, alcanzando un ROI del \textbf{300\%} al final del segundo año.
    \item \textbf{Escenario base:} Con un crecimiento moderado, los ingresos anuales proyectados son de \textbf{30,000 pesos}, logrando un ROI del \textbf{200\%} y una recuperación completa de la inversión al finalizar el periodo de incubación.
    \item \textbf{Escenario pesimista:} Una adopción lenta limita los ingresos a \textbf{18,000 pesos} anuales, con un ROI del \textbf{80\%} y una recuperación extendida hasta el tercer año.
\end{itemize}

Este análisis demuestra que, incluso en el escenario pesimista, el proyecto conserva una alta probabilidad de sostenibilidad a mediano plazo.

\section{Resultados esperados}
\subsection{Resultados esperados al final del periodo de incubación}
\begin{itemize}\setlength{\itemsep}{0pt}\setlength{\parskip}{1pt}
    \item \textbf{Lanzamiento funcional de la plataforma:} La plataforma estará operativa, con funcionalidades completas de registro, publicación de campañas, donaciones, y sistema de recompensas.
    \item \textbf{Alianzas estratégicas:} Establecimiento de acuerdos con al menos \textbf{10 fundaciones}, garantizando un flujo inicial de campañas y usuarios.
    \item \textbf{Usuarios activos:} Alcanzar un mínimo de \textbf{1,000 usuarios registrados}, promoviendo un inicio sólido para la monetización.
    \item \textbf{Generación de ingresos iniciales:} Ingresos acumulados por publicidad y comisiones superiores a \textbf{10,000 pesos}, confirmando la viabilidad económica del modelo.
\end{itemize}

\subsection{Resultados esperados a mediano y largo plazo}
\begin{itemize}\setlength{\itemsep}{0pt}\setlength{\parskip}{1pt}
    \item \textbf{Expansión de la base de usuarios:} Superar los \textbf{10,000 usuarios activos} al final del segundo año, impulsando mayores ingresos por publicidad y donaciones.
    \item \textbf{Ampliación de fundaciones participantes:} Integrar a \textbf{50 fundaciones} en los primeros tres años, diversificando las campañas disponibles y atrayendo más tráfico.
    \item \textbf{Sostenibilidad financiera:} Incrementar los ingresos anuales a más de \textbf{100,000 pesos}, garantizando la estabilidad operativa y la capacidad de reinversión.
    \item \textbf{Reconocimiento en el sector:} Consolidarse como una de las plataformas líderes en México para la gestión de campañas benéficas y voluntariado, destacando por su modelo de colaboración y gamificación.
\end{itemize}

\section{Estrategia de comercialización}

\subsection{Canales de distribución}
Los canales de distribución del proyecto estarán centrados en medios digitales para garantizar un alcance amplio y eficiente:
\begin{itemize}\setlength{\itemsep}{0pt}\setlength{\parskip}{1pt}
    \item \textbf{Sitio web propio:} La plataforma será el principal canal, permitiendo a los usuarios registrarse, explorar campañas y realizar donaciones de manera intuitiva.
    \item \textbf{Redes sociales:} Utilización de plataformas; Facebook, Instagram y Twitter para promocionar campañas, captar usuarios y mantener el compromiso de la comunidad.
    \item \textbf{Colaboraciones estratégicas:} Alianzas con fundaciones participantes a la causa y eventos relacionados con el sector para difundir la plataforma y atraer nuevos usuarios.
\end{itemize}

\subsection{Plan de marketing}

\subsubsection{Productos o servicios}
El proyecto ofrece una plataforma digital diseñada para facilitar la conexión entre usuarios y fundaciones benéficas a través de campañas de donación y voluntariado. Los servicios incluyen:
\begin{itemize}
    \item \textbf{Donaciones y pagos seguros}: A través de un sistema de pagos digital, los usuarios podrán donar de forma sencilla y segura.
    \item \textbf{Publicidad digital}: Espacios en la plataforma para que las marcas se promocionen, contribuyendo a la monetización del proyecto.
\end{itemize}

\subsubsection{Precio de venta}
El precio de venta de la plataforma será basado en comisiones sobre las donaciones realizadas:
\begin{itemize}
    \item \textbf{Comisión por donaciones}: 5\% sobre cada transacción procesada.
    \item \textbf{Tarifa por campañas destacadas}: Las fundaciones podrán pagar una tarifa para destacar sus campañas en la plataforma y llegar a más usuarios.
    \item \textbf{Publicidad digital}: Se cobrará a las marcas por el espacio publicitario dentro de la plataforma (se planea usar Google Ads).
\end{itemize}
Además, se ofrecerán opciones de suscripción como apoyo al proyecto.

\subsubsection{Objetivos}
Los principales objetivos de marketing son:
\begin{itemize}
    \item \textbf{Adquisición de usuarios}: Atraer al menos 1,000 usuarios registrados en el primer año a través de estrategias de marketing digital y alianzas con fundaciones.
    \item \textbf{Aumento del tráfico web}: Incrementar la cantidad de visitantes a la plataforma mediante campañas publicitarias y contenido relevante en redes sociales.
    \item \textbf{Monetización de la plataforma}: Generar ingresos suficientes para cubrir los costos operativos durante el periodo de incubación, con un enfoque en publicidad, comisiones y tarifas por campañas destacadas.
    \item \textbf{Posicionamiento de la marca}: Convertir la plataforma en una referencia dentro del sector benéfico, siendo reconocida por su capacidad para conectar a usuarios con causas sociales.
\end{itemize}

\subsubsection{Medios de comunicación}
Los medios de comunicación seleccionados para el marketing incluyen:
\begin{itemize}
    \item \textbf{Publicidad en redes sociales}: Utilizar plataformas como Facebook, Instagram y Twitter para promocionar la plataforma y sus campañas.
    \item \textbf{Blogs y foros}: Colaborar con influenciadores y plataformas de contenido para generar visibilidad en el sector benéfico.
    \item \textbf{Medios tradicionales}: Dependiendo de la expansión, se contemplará la inclusión de anuncios en medios impresos y televisión local para llegar a un público más amplio.
    \item \textbf{Email marketing}: Enviar boletines y promociones personalizadas a usuarios registrados y fundaciones.
\end{itemize}

\subsubsection{Marketing}
La estrategia de marketing se centrará en crear conciencia de la plataforma y fomentar la participación activa de usuarios y fundaciones. Las acciones clave incluyen:
\begin{itemize}
    \item \textbf{Marketing de contenidos}: Crear contenido relevante sobre causas sociales, testimonios de usuarios y fundaciones, e historias de éxito para generar interés y empatía.
    \item \textbf{Campañas de pago por clic (PPC)}: Utilizar anuncios pagados en Google Ads y redes sociales para captar nuevos usuarios.
    \item \textbf{Eventos y webinars}: Organizar eventos en línea para presentar la plataforma y educar sobre las ventajas de participar en campañas benéficas y de voluntariado.
\end{itemize}

\subsubsection{Marketing online}
El marketing online será el pilar central de la estrategia de adquisición de usuarios. Las tácticas incluirán:
\begin{itemize}
    \item \textbf{SEO (Search Engine Optimization)}: Optimización del sitio web para aparecer en los primeros resultados de búsqueda de Google en temas relacionados con donaciones y voluntariado.
    \item \textbf{Publicidad pagada en Google y redes sociales}: Utilizar anuncios segmentados en plataformas como Google Ads, Instagram, y Facebook para llegar a usuarios interesados en causas sociales.
    \item \textbf{Influencers y colaboraciones}: Colaborar con influenciadores en el ámbito social y de voluntariado para promocionar la plataforma y sus funcionalidades.
    \item \textbf{Marketing de afiliados}: Implementar un programa de afiliados donde los usuarios o fundaciones puedan ganar una comisión por cada nuevo usuario o donación que refieran.
\end{itemize}

\section{Conclusión}
El proyecto presenta un modelo de negocio sólido y rentable con una plataforma que conecta a los usuarios con causas sociales, facilitando donaciones y actividades de voluntariado. El análisis financiero demuestra la viabilidad del proyecto a través de indicadores clave como la TIR, VPN y ROI. Con un enfoque claro en marketing digital y estrategias de adquisición de usuarios, la plataforma está posicionada para alcanzar sus objetivos de expansión y sostenibilidad financiera. A largo plazo, se proyecta que la plataforma se consolide como un líder en el sector de campañas benéficas y voluntariado en México, generando un impacto positivo en la sociedad y garantizando la autosuficiencia financiera.


\end{document}
